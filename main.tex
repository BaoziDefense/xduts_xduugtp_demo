\documentclass{xduugtp}
\usepackage{amsmath} % 数学公式
\usepackage{graphicx} % 图片插入
\usepackage{hyperref} % 超链接
\usepackage{listings} % 代码高亮
\usepackage{xcolor} % 定义颜色
\usepackage{enumitem} % 自定义列表样式
% 加载 biblatex 包,并指定参考文献文件
\usepackage[backend=biber,style=gb7714-2015]{biblatex}

% 加载所需宏包
\usepackage{float} % 强制定位
\usepackage{caption} % 自定义标题样式
\DeclareCaptionLabelSeparator{xduugtpskip}{\hskip.75em} % 修改标题分隔符
\DeclareCaptionFont{xduugtpfont}{\rmfamily\zihao{5}} % 定义字体
\captionsetup{strut=off,labelsep=xduugtpskip,font=xduugtpfont} % 设置标题样式
\renewcommand{\figurename}{图} % 图标题名称
\renewcommand{\tablename}{表} % 表标题名称

\addbibresource{references.bib} % 指定参考文献数据库文件
\xdusetup{
style = { cjk-font = fandol, latin-font = gyre },
info = {
author = {你的名字},
major = {你的专业},
student-id = {你的学号},
supervisor = {你的指导老师},
department = {赛博土木学院},
class = {2025},
}
}
% \keys_define:nn { xdu / info }
%   {
%     class .tl_set:N = \l__xdu_class_str,
%     submit-date .tl_set:N = \l__xdu_submit_date_str,
%     sign .clist_set:N = \l__xdu_sign_clist,
%     date .clist_set:N = \l__xdu_date_clist
%   }
% \keys_define:nn { xdu / info }
%   {
%     title .tl_set:N = \l__xdu_title_tl,
%     department .tl_set:N = \l__xdu_dept_str,
%     major .tl_set:N = \l__xdu_major_str,
%     author .tl_set:N = \l__xdu_author_str,
%     supervisor .clist_set:N = \l__xdu_supv_clist,
%     supv-ent .tl_set:N = \l__xdu_supv_ent_str,
%     supervisor-enterprise .tl_set:N = \l__xdu_supv_ent_str,
%     student-id .tl_set:N = \l__xdu_student_id_str,
%     abstract .tl_set:N = \l__xdu_abstract_zh_tl,
%     abstract* .tl_set:N = \l__xdu_abstract_en_tl,
%     keywords .clist_set:N = \l__xdu_keywords_zh_clist,
%     keywords* .clist_set:N = \l__xdu_keywords_en_clist,
%     bib-resource .clist_set:N = \l__xdu_bib_file_clist,
%     appendix .clist_set:N = \l__xdu_appendix_clist,
%     acknowledgements .tl_set:N = \l__xdu_ack_tl
%   }
\begin{document}

\section{论文名称及项目来源}
本论文的名称为《基于机器学习的网络安全威胁检测》。  
项目来源:国家自然科学基金(项目编号:XXXXXX)。  

这是一个公式示例:
\[
E = mc^2
\]

\section{研究目的和意义}
\textbf{研究目的:} 提高网络安全威胁的检测精度,减少误报率。  
\textit{研究意义:} 随着物联网设备的普及,网络安全问题愈发重要。本研究将为智能安全监控系统提供技术支撑。

以下是插入超链接的示例:  
可以访问我们的研究主页:\url{https://example.com}

\section{国内外研究现状和发展趋势}
国内外研究主要集中在以下几个方向:
\begin{itemize}
    \item 使用深度学习进行流量分类;
    \item 基于统计学的异常检测;
    \item \textit{联邦学习} 在分布式环境下的应用\cite{wang2020flow}。
\end{itemize}

发展趋势包括:
\begin{enumerate}[label=(\arabic*)]
    \item 数据隐私保护技术;
    \item 轻量化模型的研究;
    \item 实时威胁检测技术\cite{yang2019federated}。
\end{enumerate}

\printbibliography

\section{主要研究内容、要解决的问题及本文的初步方案}
研究内容包括以下几点:
\begin{description}
    \item[数据预处理:] 使用标准化技术对原始数据进行清洗;
    \item[模型设计:] 基于 Transformer 架构设计一个轻量化网络模型;
    \item[性能评估:] 使用常用指标(如精准率和召回率)评估模型效果。
\end{description}

以下是插入单张图片的示例:

\begin{figure}[H] % 使用强制定位避免浮动问题
\centering
\includegraphics[width=0.6\linewidth]{whatever.jpg} % 替换为实际图片路径和名称
\caption{单张图片示例} % 图片标题
\label{fig:single} % 设置图片标签
\end{figure}

这里可以引用图片:\figurename~\ref{fig:single}。

初步方案:  
使用 \texttt{Python} 和 \texttt{TensorFlow} 实现模型,并基于开源数据集 CICIDS2017 进行训练和测试。

\section{工作的主要阶段、进度和完成时间}
计划分为以下阶段:
\begin{tabular}{|c|c|l|}
    \hline
    \textbf{阶段} & \textbf{时间} & \textbf{主要任务} \\
    \hline
    第一阶段 & 1-3 月 & 文献调研与数据集准备 \\
    第二阶段 & 4-6 月 & 模型设计与初步实验 \\
    第三阶段 & 7-9 月 & 参数优化与性能测试 \\
    第四阶段 & 10-12 月 & 论文撰写与结果提交 \\
    \hline
\end{tabular}

\section{已进行的前期准备工作}
前期准备包括以下几点:
\begin{itemize}
    \item 已调研相关文献共计 50 篇,其中 10 篇为高引用论文;
    \item 已完成数据集下载,数据量为 10GB;
    \item 已搭建开发环境,使用的工具包括:\texttt{PyCharm} 和 \texttt{Docker}。
\end{itemize}

以下是代码高亮示例:
\begin{lstlisting}[language=Python, caption=简单的Python示例代码, basicstyle=\ttfamily\small]
import numpy as np

def sigmoid(x):
    return 1 / (1 + np.exp(-x))

print(sigmoid(0))  # 输出: 0.5
\end{lstlisting}

\section{指导教师意见}
\begin{quotation}
    本研究方案合理,研究内容充实,进度安排得当。  
    建议在模型设计时进一步考虑算法的可解释性问题。
\end{quotation}

\section{学院审核意见}
\begin{quotation}
    经审核,课题符合学院的研究方向,技术路线清晰,工作计划合理。  
    同意立项,建议按计划推进。
\end{quotation}

\end{document}